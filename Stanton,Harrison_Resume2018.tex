\documentclass[12pt]{article}
\usepackage[a4paper, left=0.63in, right=0.63in, top=1in, bottom=1in]{geometry}
\usepackage{titlesec}
\usepackage{changepage}
\titlespacing\section{0pt}{12pt plus 1pt minus 1pt}{-15pt plus 1pt minus 1pt}
\titlespacing\subsection{0pt}{0pt plus 1pt minus 1pt}{0pt plus 1pt minus 1pt}
\author{Harrison Stanton}
\title{Harrison Stanton Resume}
\begin{document}
\thispagestyle{empty}
\begin{flushleft}
	\LARGE
	\textbf{HARRISON STANTON\\}
	\large
	\textbf{702-712-0826\hspace{0.5em}\vrule\hspace{0.5em}harrisonbstanton@gmail.com}
	\normalsize
	\vspace{1em}
%Objective section, to be uncommented for specific job listings/applications	
%\section*{OBJECTIVE\vspace{0.1cm\hrule}}
%To seek employment in a data driven career which allows for the usage of my developed skills and abilities in both Mathematics and Computer Science and creates an environment for the further development of these skills.
\section*{EXPERIENCE\hrule}
\subsection*{Lockheed Martin, Rotary and Mission Systems}
\begin{adjustwidth}{1.5em}{0pt}
	\textit{Software Engineer, 2021 - Current}
	\begin{adjustwidth}{1.5em}{0pt}
		Implemented several different deep convolutional neural network architectures for a novel segmentation use case.  
		Implemented 
		Working   
	\end{adjustwidth}
	\textit{Software Engineer Associate, 2018 - 2020}
	\begin{adjustwidth}{1.5em}{0pt}
		Implemented state of the art Reinforcement Learning Algorithms ( Soft-Actor Critic and Proximal Policy Optimization ) to solve novel problems.
		Worked with AWS EC2 elastic compute resources and loading docker containers with machine learning libraries.
		Programmed novel approaches to performing feature selection with reinforcement learning.
		Programmed time-series prediction with convolution and recurrent neural networks.
		
	\end{adjustwidth}
\end{adjustwidth}
\subsection*{Computer Science Capstone Course Externship}
\begin{adjustwidth}{1.5em}{0pt}
	\textit{General Electric, Reno, NV, August 2016 - May 2017}
	\begin{adjustwidth}{1.5em}{0pt}
Developed an application to classify sensor data using machine learning techniques. Displayed the data and classification results on a web page using D3.js.
	\end{adjustwidth}
\end{adjustwidth}
\subsection*{Lab Instructor}
\begin{adjustwidth}{1.5em}{0pt}
	\textit{University of Nevada, Reno, Spring 2016}
	\begin{adjustwidth}{1.5em}{0pt}
		Taught two sections of the Computer Engineering 301 lab for the University of Nevada, Reno.
	\end{adjustwidth}
\end{adjustwidth}

\section*{SKILLS\hrule}
\subsection*{Programming Languages}
\begin{adjustwidth}{1.5em}{0pt}

Experienced: C, C++, Python, Matlab

Familiar: Java

\end{adjustwidth}
\subsection*{Scripting Languages}
\begin{adjustwidth}{1.5em}{0pt}

Experienced: Bash, Csh/Tcsh

Familiar: {\LaTeX}, Javascript

\end{adjustwidth}
\subsection*{Frameworks and Tools}
\begin{adjustwidth}{1.5em}{0pt}

Experienced: PyTorch, Numpy, Gitlab(CI/CD), git, Docker/Podman

Familiar: TensorFlow, Matplotlib, QT, Boost, svn, cmake, AWS(EC2 and S3)
\end{adjustwidth}
\subsection*{Operating Systems}
\begin{adjustwidth}{1.5em}{0pt}

Experienced: Linux (Redhat, CentOS, Debian, ArchLinux, Ubuntu), Windows

\end{adjustwidth}
\section*{PROJECTS\hrule}
\subsection*{Home Server}
\begin{adjustwidth}{1.5em}{0pt}
	\textit{Personal Project, 2020 - Ongoing}
	\begin{adjustwidth}{1.5em}{0pt}
	Set up a home server for usage with personal machine learning projects and mining crypto currency. Two Supermicro 4U chassis inside a 18U rack.  
	\end{adjustwidth}
\end{adjustwidth}
\subsection*{AWS Hosted DnD Server}
\begin{adjustwidth}{1.5em}{0pt}
	\textit{Personal Project, 2021 - Ongoing}
	\begin{adjustwidth}{1.5em}{0pt}
	Set up a AWS hosted DnD server for my friends and I to DnD virtually play DnD. Set up S3 to store all the required assets, and configured the roles for S3. Software is loaded via Docker.
	\end{adjustwidth}
\end{adjustwidth}
\subsection*{GLM for Machine Learning Technique Prediction}
\begin{adjustwidth}{1.5em}{0pt}
	\textit{University of Nevada, Reno, Fall 2017 - 2018}
	\begin{adjustwidth}{1.5em}{0pt}
Programmed a machine learning classifier ensemble. A large set of datasets were then trained through the ensemble and generated higher order data.  This ensemble data was used to create a generalized linear model to predict accuracy of different classifiers based off of specific extracted attributes of the data set. For example does the distribution of the dataset have an effect on what classifiers perform well?
	\end{adjustwidth}
\end{adjustwidth}
\subsection*{Machine Learning Strategies for Solving the Bongard Problems}
\begin{adjustwidth}{1.5em}{0pt}
	\textit{University of Nevada, Reno, Fall 2016}
	\begin{adjustwidth}{1.5em}{0pt}
Constructed and trained a support vector machine and a recurrent neural network classifier on a subset of the Bongard Problems.
	\end{adjustwidth}
\end{adjustwidth}
\subsection*{Smoke Detection Prescreening in Sequential Images}
\begin{adjustwidth}{1.5em}{0pt}
	\textit{University of Nevada, Reno, Spring 2015}
	\begin{adjustwidth}{1.5em}{0pt}
Contributed to a program which identified movement patterns to locate smoke in the early stages of potential forest fires. Project was selected to be presented at ISCA CATA in 2016.  
	\end{adjustwidth}
\end{adjustwidth}

\section*{EDUCATION\hrule}
\subsection*{University of Nevada, Reno, NV}
\begin{adjustwidth}{1.5em}{0pt}
	\textit{Bachelor of Science, Discrete Mathematics, December 2017}
	\begin{adjustwidth}{1.5em}{0pt}
		Selected Coursework: Categorical Data Analysis, Statistical Machine Learning
	\end{adjustwidth}
	\textit{Bachelor of Science, Computer Science and Engineering, December 2017}
	\begin{adjustwidth}{1.5em}{0pt}
		Selected Coursework: Advanced Computer Vision, Artificial Intelligence
	\end{adjustwidth}
\end{adjustwidth}
\end{flushleft}
\end{document}